\usepackage{scrhack}

% Better support for marginnotes
% new command: \marginnote
% LaTeX standard command: \marginpar
% CTAN: http://www.ctan.org/pkg/marginnote
\usepackage{marginnote}

\usepackage{geometry}
\geometry{% siehe geometry.pdf (Figure 1)
	bottom=30mm,
	showframe=false, % For debugging: try true and see the layout frames
	margin=30mm,
	marginparsep=3mm,
	marginparwidth=20mm
}

\usepackage[T1]{fontenc}
\usepackage{calc}
\usepackage{amsmath}
\usepackage{graphicx}
\usepackage{float}
\usepackage[table]{xcolor} 
\usepackage{booktabs}
\usepackage{ragged2e}
\usepackage{listings}
\usepackage{pgfplots}
\pgfplotsset{compat=1.11}
\usepackage{natbib}
\usepackage{amssymb}
\usepackage{amsmath}
\usepackage{enumitem}
\usepackage{cancel}
\usepackage{setspace}
\usepackage{blindtext}
\usepackage{indentfirst}
\usepackage{lipsum}	
\usepackage{float}
\usepackage{graphicx}
\usepackage{verbatim}
\usepackage{mathtools}
\usepackage[portuguese]{babel}
\usepackage{comment}
\usepackage{listings}
\usepackage{xcolor}
\usepackage{animate}

\definecolor{dkgreen}{rgb}{0,0.6,0}
\definecolor{gray}{rgb}{0.5,0.5,0.5}
\definecolor{mauve}{rgb}{0.58,0,0.82}
\definecolor{codegreen}{rgb}{0,0.6,0}
\definecolor{codegray}{rgb}{0.5,0.5,0.5}
\definecolor{codepurple}{rgb}{0.58,0,0.82}
\definecolor{backcolour}{rgb}{0.95,0.95,0.92}
\definecolor[named]{myColorMainA}{RGB}{0,26,153}
\definecolor[named]{myColorMainB}{RGB}{174,49,54}
\addtokomafont{chapter}{\color{myColorMainA}}
\addtokomafont{section}{\color{myColorMainA}}
\addtokomafont{subsection}{\color{myColorMainA}}
\addtokomafont{subsubsection}{\color{myColorMainA}}
\addtokomafont{paragraph}{\color{myColorMainA}}
\addtokomafont{subparagraph}{\color{myColorMainA}}

\makeatletter
\addtokomafont{chapterentrypagenumber}{\color{myColorMainA}}
\addtokomafont{chapterentry}{\color{myColorMainA}}
\makeatother

\newcommand{\myMarginnote}[1]{%
	\marginnote{% needs marginnote package
		\ifthispageodd{\RaggedRight}{\RaggedLeft}% needs ragged2e package
		\color{myColorMainB}%
		\lineskiplimit=-\maxdimen% 
		\normalfont\sffamily\scriptsize%
		#1}%
}



\lstdefinestyle{mystyle}{
    backgroundcolor=\color{backcolour},   
    commentstyle=\color{codegreen},
    keywordstyle=\color{magenta},
    numberstyle=\tiny\color{codegray},
    stringstyle=\color{codepurple},
    basicstyle=\ttfamily\footnotesize,
    breakatwhitespace=false,         
    breaklines=true,                 
    captionpos=b,                    
    keepspaces=true,                 
    numbers=left,                    
    numbersep=5pt,                  
    showspaces=false,                
    showstringspaces=false,
    showtabs=false,                  
    tabsize=2
}
\lstset{frame=tb,
   language=python,
   aboveskip=3mm,
   belowskip=3mm,
   showstringspaces=false,
   columns=flexible,
   basicstyle={\small\ttfamily},
   numbers=none,
   numberstyle=\tiny\color{gray},
   keywordstyle=\color{blue},
   commentstyle=\color{dkgreen},
   stringstyle=\color{mauve},
   breaklines=true,
   breakatwhitespace=true
   tabsize=3
}

\tolerance 1414
\hbadness 1414
\emergencystretch 1.5em
\hfuzz 0.3pt
\widowpenalty=10000
\vfuzz \hfuzz
\raggedbottom

\usepackage[%
bookmarks, % Create bookmarks
bookmarksopen=true, % Unfold bookmatk tree in PDF viewer when document is opened
bookmarksopenlevel=1, % Level of unfolding
bookmarksnumbered=true, % Number bookmarks
hidelinks, % do not highlight hyperlinks -- looks ugly
% Ansicht beim Öffnen
pdfpagelabels=true, % See manual...
plainpages=false, % See manual...
hyperfootnotes=true, % Hyperlinks for footnotes
hyperindex=true, % Indexeinträage verweisen auf Text
]{hyperref}

\usepackage[]{blindtext}
% The custom command \myMarginnote is defined in the file: 
% 01_Preamble/HeaderFooterMarginnote.tex
\renewcommand{\blindmarkup}[1]{\myMarginnote{#1}}