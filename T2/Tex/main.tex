\documentclass%
[%
paper=a4
,fontsize=11pt % common are 10, 11 or 12
,headings=big
,numbers=noendperiod % 2.3.1 vs 2.3.1. (no dot after the last chapter number)
,twoside=true
,toc=bibliography % Bibliography appears in Table of Contents (without a number)
,toc=listof % List of Figures and List of Tables appear in Table of Contents
,version=last % Use latest version of the KOMA-Script
]%
{scrbook}
\usepackage{scrhack}

% Better support for marginnotes
% new command: \marginnote
% LaTeX standard command: \marginpar
% CTAN: http://www.ctan.org/pkg/marginnote
\usepackage{marginnote}

\usepackage{geometry}
\geometry{% siehe geometry.pdf (Figure 1)
	bottom=30mm,
	showframe=false, % For debugging: try true and see the layout frames
	margin=30mm,
	marginparsep=3mm,
	marginparwidth=20mm
}

\usepackage[T1]{fontenc}
\usepackage{calc}
\usepackage{amsmath}
\usepackage{graphicx}
\usepackage{float}
\usepackage[table]{xcolor} 
\usepackage{booktabs}
\usepackage{ragged2e}
\usepackage{listings}
\usepackage{pgfplots}
\pgfplotsset{compat=1.11}
\usepackage{natbib}
\usepackage{amssymb}
\usepackage{amsmath}
\usepackage{enumitem}
\usepackage{cancel}
\usepackage{setspace}
\usepackage{blindtext}
\usepackage{indentfirst}
\usepackage{lipsum}	
\usepackage{float}
\usepackage{graphicx}
\usepackage{verbatim}
\usepackage{mathtools}
\usepackage[portuguese]{babel}
\usepackage{comment}
\usepackage{listings}
\usepackage{xcolor}
\usepackage{animate}

\definecolor{dkgreen}{rgb}{0,0.6,0}
\definecolor{gray}{rgb}{0.5,0.5,0.5}
\definecolor{mauve}{rgb}{0.58,0,0.82}
\definecolor{codegreen}{rgb}{0,0.6,0}
\definecolor{codegray}{rgb}{0.5,0.5,0.5}
\definecolor{codepurple}{rgb}{0.58,0,0.82}
\definecolor{backcolour}{rgb}{0.95,0.95,0.92}
\definecolor[named]{myColorMainA}{RGB}{0,26,153}
\definecolor[named]{myColorMainB}{RGB}{174,49,54}
\addtokomafont{chapter}{\color{myColorMainA}}
\addtokomafont{section}{\color{myColorMainA}}
\addtokomafont{subsection}{\color{myColorMainA}}
\addtokomafont{subsubsection}{\color{myColorMainA}}
\addtokomafont{paragraph}{\color{myColorMainA}}
\addtokomafont{subparagraph}{\color{myColorMainA}}

\makeatletter
\addtokomafont{chapterentrypagenumber}{\color{myColorMainA}}
\addtokomafont{chapterentry}{\color{myColorMainA}}
\makeatother

\newcommand{\myMarginnote}[1]{%
	\marginnote{% needs marginnote package
		\ifthispageodd{\RaggedRight}{\RaggedLeft}% needs ragged2e package
		\color{myColorMainB}%
		\lineskiplimit=-\maxdimen% 
		\normalfont\sffamily\scriptsize%
		#1}%
}



\lstdefinestyle{mystyle}{
    backgroundcolor=\color{backcolour},   
    commentstyle=\color{codegreen},
    keywordstyle=\color{magenta},
    numberstyle=\tiny\color{codegray},
    stringstyle=\color{codepurple},
    basicstyle=\ttfamily\footnotesize,
    breakatwhitespace=false,         
    breaklines=true,                 
    captionpos=b,                    
    keepspaces=true,                 
    numbers=left,                    
    numbersep=5pt,                  
    showspaces=false,                
    showstringspaces=false,
    showtabs=false,                  
    tabsize=2
}
\lstset{frame=tb,
   language=python,
   aboveskip=3mm,
   belowskip=3mm,
   showstringspaces=false,
   columns=flexible,
   basicstyle={\small\ttfamily},
   numbers=none,
   numberstyle=\tiny\color{gray},
   keywordstyle=\color{blue},
   commentstyle=\color{dkgreen},
   stringstyle=\color{mauve},
   breaklines=true,
   breakatwhitespace=true
   tabsize=3
}

\tolerance 1414
\hbadness 1414
\emergencystretch 1.5em
\hfuzz 0.3pt
\widowpenalty=10000
\vfuzz \hfuzz
\raggedbottom

\usepackage[%
bookmarks, % Create bookmarks
bookmarksopen=true, % Unfold bookmatk tree in PDF viewer when document is opened
bookmarksopenlevel=1, % Level of unfolding
bookmarksnumbered=true, % Number bookmarks
hidelinks, % do not highlight hyperlinks -- looks ugly
% Ansicht beim Öffnen
pdfpagelabels=true, % See manual...
plainpages=false, % See manual...
hyperfootnotes=true, % Hyperlinks for footnotes
hyperindex=true, % Indexeinträage verweisen auf Text
]{hyperref}

\usepackage[]{blindtext}
% The custom command \myMarginnote is defined in the file: 
% 01_Preamble/HeaderFooterMarginnote.tex
\renewcommand{\blindmarkup}[1]{\myMarginnote{#1}}

\parskip=6pt 
\parindent=0pt

%\title{Trabalho de Curvas e Superfícies - A1}
%\author{Iara Cristina e Janaina Neres}
\date{\today}

\begin{document}
\thispagestyle{empty} 
	
\begin{tabular*}{0.95\textwidth}{l @{\extracolsep{\fill}} r} 
    {Curvas e Superfícies} &  \\
	Escola de Matemática Aplicada, Fundação Getulio Vargas &  \\
	Professora: Asla Medeiros de Sá &  {\today}\\
		\hline \\
\end{tabular*} 

\vspace*{0.3cm} 
\begin{center}
		{\Large Trabalho A2:  Sagemath para Superfícies}\\[5mm]
{\large Iara Cristina e Janaina Neres} \\[5mm]
		%\vspace{2mm}
	\end{center}  



  \section*{Introdução}
  Para este trabalho, escolhemos criar um caderno de estudo sobre Superfícies, bem como suas aplicações computacionais. Por meio do Sagemath, seremos capazes de compreender e visualizar as propriedades dessas estruturas matemáticas de forma interativa e prática. Apresentaremos uma variedade de exemplos e exercícios que permitirão aprofundar nosso conhecimento e fortalecer nossa compreensão sobre o assunto.
  
  \section*{Superfícies}
  Uma superfície é geralmente representada por um conjunto de equações (ou parametrizações) que descrevem as coordenadas dos pontos na superfície em termos de parâmetros livres. Essas equações ou parametrizações podem ser utilizadas para calcular as derivadas parciais, vetores normais e outras propriedades geométricas. Alguns exemplos conhecidos são: a esfera, o cilindro, o plano, o cone e o toro.
  
  \begin{lstlisting}
  u, v = var('u v')
  Toro(u,v) = [(4+(3+cos(v))*sin(u)), (4+(3+cos(v))*cos(u)), (4+sin(v))] # Figura 1
  X = parametric_plot3d(Toro, (u, 0, 2*pi), (v, 0, 2*pi), color='yellow')
  
  # S = x^2 + y^2 + z == 0 # Figura 2
  # X = implicit_plot3d(S, (x,-2,2), (y,-2,2), (z,-2,0), color='lavenderblush')
  show(X)
  \end{lstlisting}
  
  \begin{figure}[H]
    \centering
    \begin{minipage}[b]{0.4\textwidth}
      \includegraphics[width=\textwidth]{img/superficie1.png}
      \caption{Toro Parametrizado}
    \end{minipage}
    \hspace{4mm}
    \begin{minipage}[b]{0.4\textwidth}
      \includegraphics[width=\textwidth]{img/superficie2.png}
      \caption{$x^2 + y^2 + z = 0$}
    \end{minipage}
  \end{figure}
  
  %%%%%%%%%%%%%%%%%%%%%%%%%%%%%%%%%%%%%%%%%%%%%%%%%%%%
  
  \section*{Derivadas Parciais}
  \cite{de2016introduccao} \textbf{Definição:} Em coordenadas cartesianas, uma parametrização local $X: U \rightarrow S$ se escreve como
  $$
  X(u, v)=(x(u, v), y(u, v), z(u, v)),(u, v) \in U,
  $$
  e suas derivadas parciais com respeito a $u$ e a $v$ denotam-se, respectivamente, por $X_u$ e $X_v$, isto é,
  $$X_u(u, v)=\left(\frac{\partial x}{\partial u}(u, v), \frac{\partial y}{\partial u}(u, v), \frac{\partial z}{\partial u}(u, v)\right)$$
  $$X_v(u, v)=\left(\frac{\partial x}{\partial v}(u, v), \frac{\partial y}{\partial v}(u, v), \frac{\partial z}{\partial v}(u, v)\right)$$
  
  \section*{Jacobiano}
  \textbf{Definição:} Matriz Jacobiana ou Jacobiano de $f: S \subseteq \mathbb{R}^n \rightarrow \mathbb{R}^m$ no ponto $x \in S$ é a matriz $m \times n$ dada por:
  $$
  J_f(x) = \left(\begin{array}{ccc}
  \frac{\partial f_1}{\partial x_1} & \ldots & \frac{\partial f_1}{\partial x_n} \\
  \vdots & \ddots & \vdots \\
  \frac{\partial f_m}{\partial x_1} & \cdots & \frac{\partial f_m}{\partial x_n}
  \end{array}\right)
  $$
  No caso $X: S \subseteq \mathbb{R}^2 \rightarrow \mathbb{R}^3$ no ponto $p=X(q) \in S$, temos que a matriz da aplicação linear $dX_q$ nas bases canônicas de $\mathbb{R}^2$ e $\mathbb{R}^3$ é:
  
  $$dX_q = \left(\begin{array}{ll}
  \frac{\partial x}{\partial u} & \frac{\partial x}{\partial v} \\
  \frac{\partial y}{\partial u} & \frac{\partial y}{\partial v} \\
  \frac{\partial z}{\partial u} & \frac{\partial z}{\partial v}
  \end{array}\right)$$
  
  %%%%%%%%%%%%%%%%%%%%%%%%%%%%
  \section*{Superfície Parametrizada
  Regular}
  
  \textbf{Definição:} É uma aplicação $X : U \subset \mathbb{R}^2 \rightarrow \mathbb{R}^3$, com $U$ aberto de $\mathbb{R}^2$, tal que:
  \begin{enumerate}
  \item $X$ é diferenciável de classe $C^\infty$;
  \item Para todo $q(u, v) \in U$, $dX_q : \mathbb{R}^2 \rightarrow \mathbb{R}^3$ é injetora.
  \end{enumerate}
  
  Obs: O item (2) da definição garante a existência de plano tangente $T_qS$ para todos os pontos $q = (u, v) \in U \subset R^2$. Temos que as superfícies parametrizadas regulares admitem auto-intersecção, isto é, não exigem a unicidade do plano tangente em cada ponto; o que não é o caso das superfícies regulares.\\
  
   \section*{Superfície Regular} 
   
   \cite{do2016differential} \textbf{Definição:} Um subconjunto $S \subset \mathbb{R}^3$ é uma superfície regular se, para cada $p \in S$, existe uma vizinhança $V$ em $\mathbb{R}^3$ e um mapa $x: U \rightarrow V \cap S$ de um conjunto aberto $U \subset \mathbb{R}^2$ em $V \cap S \subset \mathbb{R}^3$ tal que:
  
  \begin{enumerate}
  
  \item  $\forall p \in S$, $\exists V \subseteq \mathbb{R}^3$ vizinhança de $p$ e uma aplicação $X: U \rightarrow V \cap S$ diferenciável, ou seja, é possível calcular derivadas parciais e definir vetores tangentes em cada ponto da superfície.
  
  \item  $X$ é um homeomorfismo entre $U$ e $V \cap S$, ou seja, $X$ possui uma inversa $X^{-1}: V \cap S \rightarrow U$ que é contínua.
      
      
  \item Para todo $p \in \mathbb{R}^2$, $dX_q$ é injetiva $\Leftrightarrow$ $dX_q$ tem posto 2 $\Leftrightarrow$ $X_u$ e $X_v$ LI $\Leftrightarrow X_u \times X_v \neq 0$
  \end{enumerate}
  Obs: O item (3) da definição garante a unicidade do plano tangente $T_qS$.
  \subsection*{Exemplo 1}
  
  A superfície dada por $X(u, v) = ((2\cos(u) - 1)\cos(u), (2\cos(u) - 1) \sin(u), v)$. È um exemplo de superfície parametrizada regular, mas,  não é superfície regular. 
  
  Temos que, $X$ é diferenciável de classe $C^\infty$ e $\frac{\partial X}{\partial u}(q) \times \frac{\partial X}{\partial v}(q) \neq (0, 0, 0), \forall q \in U$, garantindo a propriedade de injetividade do item 2. Porém, observa-se, que a superfície possue auto intersecção, não garantindo assim a unicidade do plano tangente em cada ponto de uma superfície regular. 
  
  \begin{figure}[H]
    \centering
    \begin{minipage}[b]{0.4\textwidth}
      \includegraphics[width=\textwidth]{img/autointersecao1.png}
      \caption{Dois planos tangentes}
    \end{minipage}
    \hspace{4mm}
    \begin{minipage}[b]{0.4\textwidth}
      \includegraphics[width=\textwidth]{img/autointersecao2.png}
      \caption{Dois planos tangentes}
    \end{minipage}
  \end{figure}
  %%%%%%%%%%%%%%%%%%%%%%%%%%%%%%%%%%%%%%%%%%%%%%%%%%%%
  
  
  \subsection*{Exemplo 2}
   Mostre que, se $f(u, v)$ é uma função real diferenciável, onde $(u, v) \in U$, aberto de $\mathbb{R}^2$, então a aplicação $X(u, v) = (u, v, f(u, v))$ é uma superfície parametrizada regular, que descreve o gráfico da função f.
  
  Por definição, $$Graf(f) = \{(x,y,z) \in \mathbb{R}^3 | (x,y) \in U \; \text{e} \; z = f(x,y)\}$$
  Então, seja $S = Graf(f)$, a aplicação $X:U \rightarrow S = S \cap \mathbb{R}^3$ dada por $X(u,v) = (u,v,f(u,v))$ é diferenciável pois $f(u,v)$ é uma função real diferenciável, cumprindo o item 1. Além disso, $X$ é uma bijeção e $X^{-1}=\left.\pi\right|_S$ é contínua, onde $\pi: \mathbb{R}^3 \longrightarrow \mathbb{R}^2, \pi(x, y, z)=(x, y)$, é a projeção sobre o plano-xy. Por último, $\mathrm{d} X_{(u, v)}: \mathbb{R}^2 \longrightarrow \mathbb{R}^3$ é injetora para todo $(u, v) \in \mathrm{U}$, pois
  $$
  \frac{\partial(x, y)}{\partial(u, v)}(u, v)=\left|\begin{array}{ll}
  1 & 0 \\
  0 & 1
  \end{array}\right| \neq 0 .
  $$
  cumprindo o item 3. Logo, $S=\operatorname{Graf}(f)$ é uma superfície regular.
  
  
  %%%%%%%%%%%%%%%%%%%%%%%%%%%%%%%%%%%%%%%
  
  \section*{Superfície de Revolução}
  \cite{de2016introduccao} \textbf{Definição:} Geometricamente, uma superfície de revolução $S$ é obtida pela rotação de uma curva regular $C$ em torno de uma reta $l$, em que $C$ e $l$ são disjuntas e contidas em um plano $\Pi \subset \mathbb{R}^3$. A curva $C$ é chamada de \textbf{geratriz} de $S$.
  
  A parametrização geral de uma superfície de revolução é dada por:
  $$
  X(u, v) = (f(v) \cos u, f(v) \sin u, g(v))
  $$
  onde \(u\) é o ângulo de rotação em torno do eixo, \(v\) é o parâmetro ao longo da curva geratriz, e \((f(v), g(v))\) são as coordenadas da curva geratriz.
  
  Para $u$ fixo temos os chamados meridianos de $S$. \\
  Para $v$ fixo temos os chamados paralelos de $S$, que são círculos, por construção.


%%%%%%%%%%%%%%%%%%%%%%%%%%%%
\section*{Cartas e Atlas}
\textbf{Definição:} Um atlas é uma coleção de mapas locais que cobrem toda a superfície.

Cada carta consiste em uma região aberta na superfície, chamada de domínio, e uma função diferenciável que associa pontos no domínio a coordenadas no plano $\mathbb{R}^2$. Essas funções permitem a parametrização local da superfície em termos das coordenadas do plano, o que facilita o estudo das propriedades geométricas da superfície.
\subsection*{Exemplo: Atlas de Esfera}

Figura 5: Atlas = $\{X_1, X_2\}$, tal que: $X_1(u,v) = (u,-\sqrt{1-(u^2+v^2)})$ e $X_2(u,v) = (u,\sqrt{1-(u^2+v^2)})$\\
Figura 6: Atlas = $\{X_3, X_4\}$, tal que: $X_3(u,v) = (u,v,-\sqrt{1-(u^2+v^2)})$ e $X_4(u,v) = (u,v,\sqrt{1-(u^2+v^2)})$
\begin{figure}[H]
  \centering
  \begin{minipage}[b]{0.4\textwidth}
    \includegraphics[width=\textwidth]{img/atlas_um.png}
    \caption{Cartas $X_1$ e $X_2$}
  \end{minipage}
  \hspace{4mm}
  \begin{minipage}[b]{0.4\textwidth}
    \includegraphics[width=\textwidth]{img/atlas_dois.png}
    \caption{Cartas $X_3$ e $X_4$}
  \end{minipage}
\end{figure}
%%%%%%%%%%%%%%%%%%%%%%%%%%%
\newpage



\section*{Mudança de Parâmetro}
\cite{do2016differential} \textbf{Definição:} Se $X$ e $Y$ são parametrizações dadas por:
$$
X(u, v)=(x(u, v), y(u, v), z(u, v)), \quad (u, v) \in U$$
$$
Y(s, t)=(x(s, t), y(s, t), z(s, t)), \quad (s, t) \in V,
$$
então a mudança de coordenadas $h$, dada por
$$
u=u(s, t), \quad v=v(s, t), \quad(s, t) \in Y^{-1}(W)
$$
tem a propriedade de que as funções $u$ e $v$ têm derivadas parciais contínuas de todas as ordens, e o mapa $h$ pode ser invertido, resultando em
$$
s =(u, v), \quad t=t(u, v), \quad(u, v) \in X^{-1}(W),
$$
onde as funções $s$ e $t$ também têm derivadas parciais de todas as ordens.

\subsection*{Exemplo}


\section*{Vetor Tangente e Plano Tangente}
\textbf{Definição:} 


\subsection*{Equações Implícitas} Seja $z=f(x, y)$ a equação de uma superfície $S$ em $\mathbb{R}^3$, se $\frac{\partial f}{\partial x}$ e $\frac{\partial f}{\partial y}$ existem em uma região em torno de um ponto $(a, b)$ e são contínuas em $(a, b)$, então o plano tangente à superfície $z=f(x, y)$ existirá no ponto $(a, b, f(a, b))$. 

Suponha que queremos uma equação do plano tangente $T$ à superfície $z=f(x, y)$ em um ponto $(a, b, f(a, b))$. Sejam $L_x$ e $L_y$ as linhas tangentes às traços da superfície nos planos $y=b$ e $x=a$: 
$$X_x = (0,1,L_x) = (0,1,\frac{\partial f}{\partial x})$$
$$X_y = (1,0,L_y) = (1,0,\frac{\partial f}{\partial y})$$

Como $T$ contém as linhas $L_x$ e $L_y$, tudo o que precisamos são vetores $X_x$ e $X_y$ que são paralelos a $L_x$ e $L_y$, respectivamente, e $n=X_x \times X_y$:
$$
\mathbf{n}=\mathbf{v}_{\mathbf{x}} \times \mathbf{v}_{\mathbf{y}}=\left|\begin{array}{ccc}
\mathbf{i} & \mathbf{j} & \mathbf{k} \\
1 & 0 & \frac{\partial f}{\partial x}(a, b) \\
0 & 1 & \frac{\partial f}{\partial y}(a, b)
\end{array}\right|
$$
Temos que $n$ é um \textbf{vetor normal} ao plano $T$. Concluímos que a equação do plano tangente para a superfície $z=f(x,y)$ no ponto $(a,b,f(a,b))$ é:
$$\frac{\partial f}{\partial x}(a, b)(x-a)+\frac{\partial f}{\partial y}(a, b)(y-b)-z+f(a, b)=0$$
$$
z = L_x(a,b) (x-a) + L_y(a,b) (y-b) + f(a,b)
$$
\section*{Curvas Coordenadas}
Dada uma parametrização $X(u,v)$:
$$X:U \subset \mathbb{R}^2 \rightarrow X(U) \subset S$$
de uma superfície regular $S$ e um ponto $p \in S$, temos que as projeções de $U$ sobre os eixos coordenados $e_u$ e $e_v$ são abertos. $I_{e_u}, I_{e_v} \subset \mathbb{R}$. Para cada ponto $(u0,v0)$, definimos as curvas coordenadas de $S$ em $X(u0,v0)$ como:

$$u \rightarrow X(u,v_0), u \in I_{e_u}$$
$$v \rightarrow X(u_0,v), u \in I_{e_v}$$

\begin{lstlisting}
X = (4+(3+cos(v))*sin(u), 4+(3+cos(v))*cos(u), 4+sin(v)) # Figura 1
# X = (u,v,u^2-v^2) Figura 2
superficie_X = parametric_plot3d(X, (u,-1,1), (v,-1,1), color='aquamarine')

u0 = 1/2 #definindo ponto u0
v0 = 1/2 #definindo ponto v0

curva_coordenada_1 = parametric_plot3d(X.subs(u=u0), (v,-1,1), color = "black", thickness = 3) #fixa v
curva_coordenada_2 = parametric_plot3d(X.subs(v=v0), (u,-1,1), color = "black", thickness = 3) #fixa u

show(superficie_X + curva_coordenada_1 + curva_coordenada_2, frame=False)    
\end{lstlisting}
\begin{figure}[H]
  \centering
  \begin{minipage}[b]{0.4\textwidth}
    \includegraphics[width=\textwidth]{img/curvas_coordenadas1.png}
    \caption{Superfície de Enneper}
  \end{minipage}
  \hspace{4mm}
  \begin{minipage}[b]{0.4\textwidth}
    \includegraphics[width=\textwidth]{img/curvas_coordenadas2.png}
    \caption{$X(x,y) = (u,v,u^2 - v^2)$}
  \end{minipage}
\end{figure}

%\section*{Superfícies de Revolução}

%\section*{Superfícies Regradas} 

\section*{Primeira Forma Fundamental}
\textbf{Definição:} O produto interno canonico de $\mathbb{R}^3$ induz em cada plano tangente $T_pS$ de uma superfície regular $S$ um produto interno. A forma quadrática 
$$\mathrm{I}_{\mathrm{p}}: \mathrm{T}_{\mathrm{p}} S \longrightarrow \mathbb{R}$$
$$\mathrm{I}_{\mathrm{p}}(w)=\langle w, w\rangle_p$$

é chamada de primeira forma fundamental da superfície regular $\mathrm{S}$ em $\mathrm{p}$. Ela permite fazer medidas sobre a superfície, como comprimento de curvas, ângulos de vetores tangentes, áreas de regiões etc.

Para obter a primeira forma fundamental de uma superfície parametrizada $$X(u,v) = (x(u,v),y(u,v),z(u,v)):U\subset \mathbb{R}^2 \rightarrow X(U) \subset S$$

Primeiro devemos calcular as derivadas direcionais em $p=X(u,v)$.
$$X_u(u,v) = (\frac{\partial}{\partial u} x(u,v), \frac{\partial}{\partial u} y(u,v), \frac{\partial}{\partial u} z(u,v))$$
$$X_v(u,v) = (\frac{\partial}{\partial v} x(u,v), \frac{\partial}{\partial v} y(u,v), \frac{\partial}{\partial v} z(u,v))$$

E calcular os coeficientes da primeira formula fundamental:
$$E = X_u(u,v) \cdot X_v(u,v)$$
$$F = X_u(u,v) \cdot X_v(u,v)$$
$$G = X_v(u,v) \cdot X_v(u,v)$$

Através dela, encontramos a matriz $I_{X(u,v)}$ com respeito a base $\{X_u(u,v), X_v(u,v)\}$ de $T_{X(u,v)}S$  
$$
\begin{bmatrix}
E(u,v) & F(u,v)\\
F(u,v) & G(u,v)
\end{bmatrix}$$
E daí,
$$\begin{bmatrix}
a & b
\end{bmatrix}\begin{bmatrix}
E & F\\
F & G
\end{bmatrix} \begin{bmatrix}
a \\
b
\end{bmatrix}$$
$$I_{X(u,v)}(w) = a^2 E(u,v) + 2ab F(u,v) + b^2 G(u,v)$$

\subsection*{Exemplo: Primeira Forma Fundamental do Toro}
O toro é a imagem do domínio do plano $D = [0, 2\pi] \times [0, 2\pi]$ pela parametrização: $$\varphi(u, v) = ((R + r \cos v) \cos u, (R + r \cos v) \sin u, r \sin v)$$
Que é uma superfície de revolução com $f(v) = R + r \cos v$ e $g(v) = r \sin v$.

Os coeficientes da primeira forma fundamental desta parametrização são 
\begin{align*}
E(u, v) &= (R + r \cos v)^2, \\
F(u, v) &= 0, \\
G(u, v) &= (-r \sin v)^2 + (r \cos v)^2 = r^2.
\end{align*}

\subsubsection*{Função em Sagemath}
\begin{lstlisting}
def obter_coef_primeira_forma(X, u, v):
    x_u = diff(X, u)
    x_v = diff(X, v)
    
    # Primeira Forma Fundamental
    E = x_u.dot_product(x_u)
    F = x_u.dot_product(x_v)
    G = x_v.dot_product(x_v)
    
    E = E.full_simplify().canonicalize_radical()
    F = F.full_simplify().canonicalize_radical()
    G = G.full_simplify().canonicalize_radical()
    
    return E,F,G    
\end{lstlisting}
%%%%%%%%%%%%%%%%%%%%%%%%%%%%%%%%%%%%%%%%%%%%%%%%%%%%%%%%%%%%%%%%%%
\section*{Área de Superfície}

A área de uma região pode ser obtida pelos coeficientes da Primeira Forma Fundamental:
$$A = \int_{X^{-1}(R)} \sqrt{E \cdot G - F^2} du \; dv$$

\subsection*{Exemplo: Área do Toro}
Continuando o exemplo anterior, temos que:
\begin{align*}
E(u, v) &= (R + r \cos v)^2, \\
F(u, v) &= 0, \\
G(u, v) &= (-r \sin v)^2 + (r \cos v)^2 = r^2.
\end{align*}

$\sqrt{EG - F^2} = \sqrt{r(R + r \cos v)}$ , e a área do toro é
\begin{align*}
A = \int_{0}^{2\pi} \int_{0}^{2\pi} (R + r \cos v) \, du \, dv \\
= 2\pi r \int_{0}^{2\pi} (R + r \cos v) \, dv \\
= 2\pi r \left[Rv + r \sin v\right]_{v=0}^{v=2\pi} \\
= 4\pi^2 rR.
\end{align*}
% \url{http://150.164.25.15/~rodney/notas_de_aula/geometria_diferencial.pdf}
\subsubsection*{Função em Sagemath}
\begin{lstlisting}
def calcular_area(X, u, v, Intervalo_u, Intervalo_v):
    E, F, G = obter_coef_primeira_forma(X, u, v)
    A = integrate(integrate(sqrt(E * G - F^2), u, Intervalo_u[0], Intervalo_u[1]), v, Intervalo_v[0], Intervalo_v[1])
    
    return A.full_simplify()
\end{lstlisting}
%%%%%%%%%%%%%%%%%%%%%%%%%%%%%%%%%%%%%%%%%%%%%%%%%%%%%%%%%%%%%%%%%%
\section*{Vetor Unitário}
Seja $X_u(u,v)$ e $X_v(u,v)$ as derivadas parciais de uma parametrização $X$, consideramos o vetor normal unitário em um ponto $X(u,v)$:
$$n(u,v) = \frac{X_u(u,v) \cdot X_v(u,v)}{||X_u(u,v) \cdot X_v(u,v)||}$$
\subsubsection*{Função em Sagemath}
\begin{lstlisting}
def obter_normal_unitario(X, u, v):
    x_u = diff(X, u)
    x_v = diff(X, v)
    
    normal = x_u.cross_product(x_v)   
    n = normal / normal.norm()
    n = n.apply_map(lambda x: x.simplify_full().canonicalize_radical())
    
    return n
\end{lstlisting}
\section*{Aplicação Normal de Gauss}

Seja $S \subset \mathbb{R}^3$ uma superfície regular orientável e $N$ um campo normal unitário diferenciável em $S$, isto é, $|N(p)| = 1$ para todo $p \in S$, ou seja, $N$ toma valores na esfera unitária centrada na origem $S^2$. 
Nessas condições, $N: S \rightarrow S^2$ é chamada de \textbf{aplicação normal de Gauss} de $S$. 


%%%%%%%%%%%%%%%%%%%%%%%%%%%%%%%%%%%%%%%%%%%%%%%%%%%%%


\section*{Segunda Forma Fundamental}
Para obter a primeira forma fundamental de uma superfície parametrizada $$X(u,v) = (x(u,v),y(u,v),z(u,v)):U\subset \mathbb{R}^2 \rightarrow X(U) \subset S$$

Primeiro devemos calcular as derivadas direcionais em $p=X(u,v)$.

$$X_u(u,v) = (\frac{\partial}{\partial u} x(u,v), \frac{\partial}{\partial u} y(u,v), \frac{\partial}{\partial u} z(u,v))$$

$$X_v(u,v) = (\frac{\partial}{\partial v} x(u,v), \frac{\partial}{\partial v} y(u,v), \frac{\partial}{\partial v} z(u,v))$$

Agora vamos calcular as segundas derivadas:
$$X_{uu}(u,v) = \frac{\partial}{\partial u}X_u(u,v)$$
$$X_{uv}(u,v) = \frac{\partial}{\partial u}X_u(u,v)$$
$$X_{vv}(u,v) = \frac{\partial}{\partial u}X_v(u,v)$$

Calculamos o Vetor Unitário Normal à Superfície:

$$n = \frac{X_u(u,v) \cdot X_v(u,v)}{||X_u(u,v) \cdot X_v(u,v)||}$$

E encontramos os coeficientes da Segunda Forma Fundamental:

$$L = n(u,v) \cdot X_{uu}(u,v)$$
$$M = n(u,v) \cdot X_{uv}(u,v)$$
$$N = n(u,v) \cdot X_{vv}(u,v)$$

Assim como na Primeira Forma, encontramos a matriz $II_{X(u,v)}$ de $T_{X(u,v)}S$  
$$\begin{bmatrix}
L(u,v) & M(u,v)\\
M(u,v) & N(u,v)
\end{bmatrix}$$

E daí,
$$\begin{bmatrix}
a & b
\end{bmatrix} \begin{bmatrix}
L & M\\
M & N
\end{bmatrix} \begin{bmatrix}
a \\
b
\end{bmatrix}$$
$$II_{X(u,v)}(w) = a^2 L(u,v) + 2ab M(u,v) + b^2 N(u,v)$$

\subsection*{Exemplo:}

\subsubsection*{Função em Sagemath}
\begin{lstlisting}
def coef_segunda_forma(X, u,v):
    # Derivadas de Primeira e Segunda Ordem
    x_u = derivative(X, u)
    x_v = derivative(X, v)
    
    x_uu = derivative(x_u, u)
    x_uv = derivative(x_u, v)
    x_vv = derivative(x_v, v)
    
    # Vetor Normal Unitario a Superficie
    vetor_N = obter_normal_unitario(X, u, v)
    
    # Segunda Forma Fundamental
    L = vetor_N.dot_product(x_uu)
    M = vetor_N.dot_product(x_uv)
    N = vetor_N.dot_product(x_vv)
    
    L = L.full_simplify().canonicalize_radical()
    M = M.full_simplify().canonicalize_radical()
    N = N.full_simplify().canonicalize_radical()
    
    return L, M, N 
\end{lstlisting}
\section*{Aplicações Lineares Auto-Adjuntas e Formas Quadráticas}
\textbf{Definição:} Um operador linear $A: V \rightarrow V$ é auto-adjunto se:
$$<Av, w> \; = \; <v, Aw> \quad \forall \; v,w \in V$$
\begin{itemize}
    \item $V$: espaço vetorial real
\end{itemize}

$$A \; \text{simétrica} \leftrightarrow A \; \text{auto-adjunta}$$

%%%%%%%%%%%%%%%%%%%%%%%%%%%%%%%%%%%%%%%%%%%%%%%%%%%%%
\section*{Curvatura de Superfícies}

Seja $S$ uma superfície regular orientável com orientação $N$. Vimos que $-dN_p$ é diagonalizável, ou seja, existem uma base ortonormal ${w_1, w_2}$ associada aos autovalores $k_1$ e $k_2$, ou seja:
$-dN_pw_1 = k_1w_1$ e $-dN_pw_2 = k_2w_2$.

Definimos então:

\subsection*{Curvaturas Principais}

O máximo e o mínimo da curvatura normal em um determinado ponto $p$ em uma superfície são chamados de curvaturas principais $\mathbf{k}_1(p)$ e  $\mathbf{k}_2(p)$, respectivamente.
 
Se $k_1(p) = k_2(p) = 0$, $p$ é um ponto planar;\\
Se $k_1(p) = k_2(p)$, $p$ é um ponto umbílico.

Obs: (Dado um ponto p em uma superfície, as curvaturas principais $k_1(p)$ e $k_2(p)$ são determinadas pelos autovalores da matriz da segundo forma fundamental no ponto $p$)
\subsubsection*{Função em Sagemath}
\begin{lstlisting}
def curvaturas_principais(X, u, v):
    dNp = matriz_forma(X, u, v)
    k1, k2 = dNp.eigenvalues()
    
    return k1, k2
\end{lstlisting}

\subsection*{Direções principais}

As direções principais  $\mathbf{w}_1(p)$ e $\mathbf{w}_2(p)$ são os autovetores  associados aos autovalores $k_1(p)$ e $k_2(p)$, respectivamente, da matriz $-dN_p$.
No ponto $p$ da superfície $S$, o vetor $w_1(p)$ aponta na direção de máxima curvatura, enquanto $w_2(p)$ aponta na direção de mínima curvatura.

\subsection*{Curvatura de Gauss}
A curvatura de Gauss é uma medida da curvatura intrínseca de uma superfície em um determinado ponto. E pode ser representada pela determinante da matriz $-dN_p$, ou por: 
$$\mathbf{K(p) = k_1 \cdot k_2}$$

Se $K(p) > 0$ (ambos $k_1(p)$ e $k_2(p)$ têm o mesmo sinal), $p$ é um ponto elíptico;\\
Se $K(p) < 0$ (ambos $k_1(p)$ e $k_2(p)$ têm sinais opostos), $p$ é um ponto hiperbólico;\\
Se $K(p) = 0$ (com $-dN_p \neq 0$), $p$ é um ponto parabólico.\\

\subsubsection*{Função em Sagemath}
\begin{lstlisting}
def curvatura_gauss(X, u, v):
    dNp = matriz_forma(X, u, v)
    curv_gauss = dNp.determinant().full_simplify()
    # k1, k2 = curvaturas_principais(X,u,v)
    # curv_gauss = k1.dot_product(k2)
    
    return curv_gauss
\end{lstlisting}
\subsection*{Curvatura Média}

A curvatura média de $S$ em $p$ é definida como sendo:

$$\mathbf{H(p) = \frac{1}{2} \operatorname{tr}(-dN_p) = \frac{k_1(p) + k_2(p)}{2}}$$

\subsubsection*{Função em Sagemath}
\begin{lstlisting}
def curvatura_media(X, u, v):
    k1, k2 = curvaturas_principais(X, u, v)
    curv_media = (k1 + k2) / 2
    
    return curv_media
\end{lstlisting}


\section*{Bônus: Transição entre Superfícies}
Para criar uma transição suave entre duas superfícies no Sagemath, existem alguns métodos. Seja $X = (x_1,x_2,x_3)$ e $Y = (y_1,y_2,y_3)$ as parametrizações das superfícies, podemos criar a parametrização de transição $T$ sendo:
$$T = (cos(a)*x_1 + sin(a)*y_1, \; cos(a)*x_2 + sin(a)y_2, \; cos(a)*x_3 + sin(a)*y_3)$$

Daí, quando $a = 0$, teremos a superfície Y e quando $a = 2\pi$ (ou $-2\pi$), teremos a superfície X. Então basta aplicar a função $animate()$ com $a$ variando entre 0 e $2\pi$.

Créditos: Pedro García

\begin{lstlisting}
a,u,v,z  = var('a u v z')

# Helicoide
Helicoid = ((sinh(u)*cos(v), sinh(u)*sin(v), v))
# Catenoide
Catenoide = ((-cosh(u)*sin(v), cosh(u)*cos(v), u))
# Transicao
X(u,v) = (cos(a)*Helicoid[0] + sin(a)*Catenoid[0], cos(a)*Helicoid[1] + sin(a)*Catenoid[1], cos(a)*Helicoid[2] + sin(a)*Catenoid[2])

figura = parametric_plot3d(X.subs(a=i), (u, -4,4), (v, -4, 4)) for i in sxrange(-2*pi, 0,0.3)
animacao = animate(figura)
show(animacao)
\end{lstlisting}
\begin{figure}[H]
  \centering
  \begin{minipage}[b]{0.4\textwidth}
    \animategraphics[autoplay,loop,width=5cm]{4}{gif/transicao_catenoide-}{0}{21}
    \caption{Transição suave entre Helicoide e Catenoide} 
  \end{minipage}
  \hspace{5mm}
  \begin{minipage}[b]{0.4\textwidth}
    \animategraphics[autoplay,loop,width=5cm]{4}{gif/transicao_hiperboloide-}{0}{26}
    \caption{Transição suave entre Helicoide e Hiperboloide} 
  \end{minipage}
\end{figure}

Analogamente, há a alternativa de utilizar $(1-a)$ em vez de $cos(a)$ e $a$ em vez de $sen(a)$. 

$$T = ((1-a)*x_1 + a*y_1, \; (1-a)*x_2 + ay_2, \; (1-a)*x_3 + a*y_3)$$

Daí, quando $a$ = 0, teremos a superfície $X$ e quando $a = 1$, teremos a superfície $Y$. Então basta aplicar a função $animate()$ com $a$ variando entre $0$ e $1$.

\begin{lstlisting}
# Helicoide
Helicoid = ((v*cos(u), v*sin(u), u))
# Hiperboloide de uma folha
Hiperboloid = ((cos(u) - v*sin(u), sin(u) + v*cos(u), v))
# Transicao
X(u,v) = ((1-a)*Helicoid[0] + a*Hiperboloid[0], (1-a)*Helicoid[1] + a*Hiperboloid[1], (1-a)*Helicoid[2] + a*Hiperboloid[2])

figura = parametric_plot3d(X.subs(a=i), (u, -2*pi,0), (v, -2, 2)) for i in sxrange(0, 1,0.1)
animacao = animate(figura)
show(animacao)
\end{lstlisting}
\begin{figure}[H]
  \centering
  \begin{minipage}[b]{0.4\textwidth}
    \animategraphics[autoplay,loop,width=5cm]{4}{gif/transicao2_catenoide-}{0}{9}
    \caption{Transicao suave entre Helicoide e Catenoide} 
  \end{minipage}
  \hspace{5mm}
  \begin{minipage}[b]{0.4\textwidth}
    \animategraphics[autoplay,loop,width=5cm]{4}{gif/transicao2_hiperboloide-}{0}{10}
    \caption{Transição suave entre Helicoide e Hiperboloide} 
  \end{minipage}
\end{figure}
%
\bibliographystyle{plain}
\bibliography{bibliography.bib}

\end{document}