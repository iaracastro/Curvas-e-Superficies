\indent Para este trabalho, escolhemos trabalhar com Curvas Offset matematicamente e computacionalmente.  A importância de Curvas offset se dá pelas aplicações em diversas áreas, como usinagem, modelagem geométrica, computação gráfica e robótica. Podemos citar alguns exemplos: criar caminhos para ferramentas de corte, criar superfícies com certa espessura, planejar trajetórias de movimento e definir regiões de tolerância em peças mecânicas. 

O estudo das propriedades geométricas e topológicas de curvas offset pode levar a novos insights matemáticos que contribuem para o desenvolvimento de novas técnicas para sua manipulação e uso em aplicações práticas. Vamos começar definindo alguns conceitos de curvas que serão utilizados e discutidos:\\

\noindent $\bullet$ Para uma curva parametrizada regular $\alpha:I \rightarrow \mathbb{R}^2$ representada por $\alpha(t) = (x(t),y(t))$, sua derivada é $\alpha'(t) = (x'(t), y'(t))$
\vspace{5pt}

\noindent $\bullet$ Para uma curva parametrizada  regular $\alpha:I \rightarrow \mathbb{R}^3$ representada por $\alpha(t) = (x(t),y(t),z(t))$, sua derivada é $\alpha'(t) = (x'(t), y'(t),z'(t))$
\vspace{5pt}

\noindent $\bullet$ Uma curva é regular, se $\alpha'(t) \neq 0 \; \forall \; t \in I$
\vspace{5pt}

\noindent $\bullet$ Uma curva é unit speed, se $||\alpha'(t)|| = 1$
\vspace{5pt}

\noindent $\bullet$ O vetor tangente $T$ é definido por $\frac{\alpha}{||\alpha'(t)||}$ no caso geral.
\vspace{5pt}

\noindent $\bullet$ O vetor normal $N$ é definido por $\frac{\alpha''(t)}{||\alpha''(t)||}$, ou também, $\frac{T'(t)}{|T'(t)|}$.

\vspace{5pt}
\noindent $\bullet$ O vetor binormal $B$ é definido por $T(t) \times N(t)$