Após investigar sobre o caso de curvas offset em funções implícitas, concluímos que geralmente a representação analítica de uma curva paralela de uma curva implícita não é possível. Mas existe uma forma de representar a distância orientada analiticamente. Para isso calculamos:

\noindent $\bullet$ Grad: O gradiente da função C.\\
\noindent $\bullet$ n: O gradiente da função C normalizado.\\
\noindent $\bullet$ N(x,y): Produto de cada componente de Grad com seu respectivo componente em n. \\
\noindent $\bullet$ Offset: Curva offset representada por $O(x,y) = C(x,y) + d*N(x,y)$
\begin{figure}[H]
  \centering
  \begin{minipage}[b]{0.45\textwidth}
    \includegraphics[width=\textwidth]{img/implicit1.png}
    \caption{$F(x,y) = sen^4(x) + cos^4(y) - 1$}
  \end{minipage}
  \hfill
  \begin{minipage}[b]{0.45\textwidth}
    \includegraphics[width=\textwidth]{img/implicit2.png}
    \caption{$F(x,y) = sen(x)^2 + cos(x)^2 - 1$}
  \end{minipage}
\end{figure}

Apesar de obter os resultados desejados, não foi possível restringir a distância $d$ como gostaríamos, e nem sempre obteremos curvas paralelas.
Para mais detalhes \cite{enwiki:1124144545}.