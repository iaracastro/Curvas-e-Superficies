As curvas offset são importantes em "Numericall Controlled Machining" ou "Usinagem Controlada Numericamente", onde definem a forma do corte feito por uma ferramenta de corte redonda de uma máquina de dois eixos que é transladada da trajetória do cortador por uma distância constante na direção normal à trajetória do cortador em cada ponto. Uma dessas ferramentas de corte é a "Ball-End Nose Cutter", que é usada em Controle Numérico Computadorizado (CNC) que tem uma ponta esférica para criar superfícies curvas em objetos tridimensionais. Segue a função que gera uma animação para alguns exemplos: 

\begin{comment}
\begin{lstlisting}[language=Python]
def get_offset_curve(alpha, d):
    #---- INPUT ------
    # alpha: curva
    # d: distancia entre a curva original e offset
    #-----------------
    
    d_alpha = diff(alpha, t) # calcula o vetor tangente
    T = d_alpha/d_alpha.norm() # normaliza
    N = diff(T, t)/diff(T, t).norm() # calcula o vetor normal
    offset = alpha + d*N # aplica a formula da curva offset
    
    return offset
\end{lstlisting}  
\end{comment}
\begin{lstlisting}[language=Python]    
def ball_end(C, d, t_range, x_range, y_range):
    T = diff(C, t) /diff(C, t).norm() # vetor tangente
    N = vector((- T[1], T[0])) # vetor normal
    Od = C + d*N # curva offset
    
    frames = [parametric_plot(Od, (t, t_range[0], t_range[1]), color="black") +
          parametric_plot(C, (t, t_range[0], t_range[1]), color = "blue") +
          plot(N.subs(t=i)*2*d, start = C.subs(t=i) ,  color = "red" ) +
          plot(circle(Od.subs(t=i), d, color = "blue")) for i in srange(t_range[0], t_range[1], 0.5)]
    return animate(frames,figsize=10,xmin=x_range[0], xmax=x_range[1], ymin=y_range[0], ymax=y_range[1])
\end{lstlisting}

\begin{figure}[H]
    \centering
    \animategraphics[autoplay,loop,width=7cm]{6}{gif/rec-}{0}{28}
    \caption{"ball-end curve cutter" ao longo do offset de $C(t) = (t,t^2)$} 
\end{figure}
 
\begin{figure}[H]
    \centering
    \animategraphics[autoplay,loop,width=8cm]{4}{gif2/rec2-}{0}{13}
    \caption{"ball-end curve cutter" ao longo do offset de $C(t) = (2\cos(t),2\sin(t))$}
\end{figure}