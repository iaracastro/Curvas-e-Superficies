Os exercícios resolvidos são propostos na página 108 do livro \cite{hartmann2003geometry} "Applied Geometry for Computer Graphics and CAD" do Duncan Marsh.

Começamos criando duas funções, a primeira retorna a equação da curva offset, a partir da curva original e o valor "d", enquanto a segunda gera os plots de múltiplas curvas offsets para vários valores de $d$.

\begin{lstlisting}[language=Python]
def get_offset_curve(alpha, d):
    #---- INPUT ------
    # alpha: curva
    # d: distancia entre a curva original e offset
    #-----------------
    C(t) = alpha
    C_d(t) = diff(C(t), t)
    v_n(t) = (- C_d(t)[1], C_d(t)[0]) # vetor normal 
    n(t) = v_n(t) / v_n(t).norm() # vetor normal unitario
    offset = C(t) + d*n(t) # aplica a formula da curva offset
    
    return offset
\end{lstlisting}

\begin{lstlisting}[language=Python]
def plot_multiple_d(alpha, d, k, intervalo, size):   
    figures = []
    curve = figures.append(parametric_plot(alpha, (t, intervalo[0], intervalo[1]), figsize=size, color="blue"))
    for i in range(1,k+1):
        d_k = d*i
        positive_offset = get_offset_curve(alpha, d_k)
        negative_offset = get_offset_curve(alpha, -d_k)
        
        if k < 4:
            figures.append(parametric_plot((positive_offset), (t, intervalo[0], intervalo[1]), figsize=size, color="green", legend_label=f"d ={float(str(d_k).rstrip('0').rstrip('.'))}"))   
            figures.append(parametric_plot((negative_offset), (t, intervalo[0], intervalo[1]), figsize=size, color="red", legend_label=f"d ={float(str(-d_k).rstrip('0').rstrip('.'))}"))
        else:
            figures.append(parametric_plot((positive_offset), (t, intervalo[0], intervalo[1]), figsize=size, color="green"))   
            figures.append(parametric_plot((negative_offset), (t, intervalo[0], intervalo[1]), figsize=size, color="red"))
    return sum(figures)
\end{lstlisting}

\subsection*{Questões}
\noindent 5.15. Determine a offset de $\mathbf{C}(t)=\left(1-3 t+3 t^2, 3 t^2-2 t^3\right)$ em uma distância d. Plote a curva e sua offcet em uma distância $d=1$.\\

\noindent 5.16. Determine as offsets à uma distance $d$ para as seguintes curvas:\\
(a) $(c \cosh (t / c), t)$\\
(b) $\left(e^{b t} \cos t, e^{b t} \sin t\right)$.\\
(c) $(\cos t+t \sin t, \sin t-t \cos t)$.\\


\noindent 5.17. Mostre que a offset em uma distância $d$ de um círculo de raio $r$ é um círculo de raio $r+d$.

\subsection*{Soluções}
\subsubsection*{5.15}
\begin{figure}[H]
    \centering
    \includegraphics[width=7cm]{img/5-15-1.1.png}
    \caption{Exercício 5.15, d = 1}
\end{figure}

\begin{figure}[H]
    \centering
    \includegraphics[width=8cm]{img/5-15-1.2.png}
    \caption{Exercício 5.15, outros valores de d}
\end{figure}

\subsubsection*{5.16, a)}
\begin{figure}[H]
    \centering
    \includegraphics[width=7cm]{img/5-16-a.png}
    \caption{$C(t) = (c \cosh (t / c), t)$, $c = e$}
\end{figure}

\subsubsection*{b)}
\begin{figure}[H]
    \centering
    \includegraphics[scale=0.5]{img/5-16-b.png}
    \caption{$C(t) = \left(e^{b t} \cos t, e^{b t} \sin t\right)$}
\end{figure}

\subsubsection*{c)}
\begin{figure}[H]
    \centering
    \includegraphics[scale=0.5]{img/5-16-c.png}
    \caption{$(\cos t+t \sin t, \sin t-t \cos t)$}
\end{figure}

\subsubsection*{5.17}
Para provar que a offset em uma distância $d$ de um círculo de raio $r$ é um círculo de raio $r+d$, seja a parametrização de um círculo de raio $r$, $C(t) = (r*cos(t),r*sin(t))$, calculamos a expressão da curva offset e mostramos que é a parametrização de um círculo de raio $(r+d)$
\begin{figure}[H]
    \centering
    \includegraphics[scale=0.6]{img/5-17-explanation.png}
\end{figure}

\begin{figure}[H]
    \centering
    \includegraphics[scale=0.5]{img/5-17.png}
    \caption{Exercício 5.17: para vários valores de d}
\end{figure}